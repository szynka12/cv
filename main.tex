\documentclass[a4paper, 10pt]{article}

\usepackage{mycvstyle}
\pagestyle{plain}

\name{Wojciech Sadowski}
\address{Al. Komisji Edukacji Narodowej 94/7}{02-777, Warsaw, Poland}
\phone{+48 506 741 115}
\mail{wojciech1sadowski@gmail.com}
\github{github.com/szynka12}
\linkedin{linkedin.com/in/w1sadowski/}
\personal{Ambitious and hard-working engineer interested in numerical analysis and oriented on applying it to real-world problems. Always open for new challenges and opportunities to broaden my set of skills. I enjoy working in interdisciplinary teams, with people who share similar ideas and attitude to technology.}
\def \WUT {Warsaw University of Technology}
\def \MEiL {Faculty of Power and Aeronautical Engineering}

\newcommand{\skill}[2]{%
  {#1} \hfill \comment{{#2}}
}

\newenvironment{skillhead}[1]{%
  \textbf{{#1}:}\par
}
{
  \vspace{0.5em}
}

\newcommand{\achi}[2]{%
  {#1} \comment{(\textit{#2})}
  \vspace{0.2em}
}

% \raggedbottom

\def \addspace {0.0em}
\usepackage[official]{eurosym}

\begin{document}
%\info{}
\infolr{}

\vspace{0.4em}

\leftright{%
  \begin{cvsection}{Work experience}
    \begin{entry}{Project leader}{\hfill \comment{\normalsize{\textit{since Sept. 2019}}}}
      The Institute of Applied Research of the Warsaw University of Technology, QuickerSim Ltd.\par
      \begin{itemize}
        \item Leading a joint research project (budget: approx. 80.000\euro{}) concerning the increase of accuracy of 3D printing in SLS technology by the means of coupled thermal and structural analyses.
        \item Coordinating the team consisting of three engineers.
        \item Running CFD simulations in commercial software and writing specialised coupled thermal/structural numerical code for the project.
      \end{itemize}
    \end{entry}
    \vspace{\addspace}
  \begin{entry}{CFD Engineer / Developer}{\hfill \comment{\normalsize{\textit{since Dec. 2017}}}}
      QuickerSim Ltd.
      \begin{itemize}
        \item Implemented Finite Element Method solver employing geometrically exact beam elements for structural analyses with small or large deformations and linear or non-linear buckling simulations.
          Conducted industrial-grade analysis of pressure filters with the aforementioned solver.
        \item Implemented various RANS turbulence models (e.g.\ Chien \(k-\varepsilon\), Wilcox \(k-\omega\)) in QuickerSim in-house Finite Element Method based CFD solver adopted in industry and academia.
          Validated implemented models against established theoretical and experimental results.
        \item Conducted analysis of flow-induced noise using Large Eddy Simulation by means of Lattice-Boltzmann method for research and development of hearing aids.
        \item Simulated blood flow for artificial organ research and development.
        \item Implemented various stabilisation methods for CFD in Finite Element Method framework (e.g. Flux Corrected Transport). 
      \end{itemize}
    \end{entry}
    \vspace{\addspace}
    \begin{entry}{Intern}{\hfill \comment{\normalsize{\textit{Sept. 2017 - Dec. 2017}}}}
      Bosch, Product Life Management
      \begin{itemize}
        \item Provided technical support for CAD and CAM software users in various Bosch departments around the globe.
        \item Beta-tested new versions of CAD/CAM software.
        \item Administrated PLM data.
      \end{itemize}
    \end{entry}
    \vspace{\addspace}
    \begin{entry}{Vehicle Dynamics Engineer}{\hfill \comment{\normalsize{\textit{Oct. 2016 - Sept. 2017}}}}
      Hyper Poland University Team
      \begin{itemize}
        \item Designed and constructed first polish Hyperloop pod prototype.
        \item Coordinated mechanical assembly of the prototype.
        \item Implemented Hyperloop prototype dynamic model in Matlab and Simulink.
        \item Designed and manufactured prototype lateral stabilizers.
      \end{itemize}
    \end{entry}
    \vspace{\addspace}
    \begin{entry}{Mechanical Engineer}{\hfill \comment{\normalsize{\textit{Oct. 2015 - Jun. 2018}}}}
      Students Association of Vehicle Aerodynamics
      \begin{itemize}
        \item Constructed extremely fuel efficient vehicles Kropelka 2.0 and PAKS.
        \item Coordinated mechanical team (5 people) of Kropelka 2.0 project.
        \item Improved and redesigned the drivetrain of Kropelka 2.0.
      \end{itemize}
    \end{entry}
  \end{cvsection}
}{%
  \begin{cvsection}{Education}
    \begin{entry}{M. Eng.}{Mechanical Engineering}
      \\%
      \WUT, \MEiL  \par
      %
      \comment{\textit{Feb. 2018 - Sept. 2019}}
      \comment{\textbf{Thesis:} \textit{Assessment of an algebraic intermittency model for separation-induced transition}}
      \comment{\textbf{GPA:} \textit{4.33}\hfill (scale: \textit{2.0}-\textit{5.0}; higher is better)}
      %
    \end{entry}
    %
    \begin{entry}{B. Eng.}{Robotics,}
      \\%
      \WUT, \MEiL  \par
      %
    \comment{\textit{Sept. 2014 - Feb. 2018}}
      \comment{\textbf{Thesis:} \textit{Trajectory planning and obstacle avoidance in cluttered environment}}
    \comment{\textbf{GPA:} \textit{4.07}\hfill (scale: \textit{2.0}-\textit{5.0}; higher is better)}
      %
    \end{entry}
  \end{cvsection}
  
  \begin{cvsection}{Achievements}
    \achi{Ministry of Science and Higher Education Scholarship for Scientific Achievements}{March 2018}
    \achi{Finalist of Hyperloop Pod Competition II}{Los Angeles, August 2017}
    \achi{3rd place, Kropelka 2.0 project, Shell Eco-Marathon Challenger}{Le Mans, 2018}
    \achi{2nd place, PAKS project, Shell Eco-Marathon Challenger}{Le Mans, 2016}
    \vspace{0.3em}
  \end{cvsection}

  \begin{cvsection}{Skills}
  \begin{skillhead}{Engineering and Science}
    \skill{Turbulence modelling}{very good}
    \skill{CFD}{good}
    \skill{Finite Element Method}{very good}
  \end{skillhead}
  \begin{skillhead}{Programming}
    \skill{C/C++}{intermediate}
    \skill{Matlab, Simulink}{very good}
    \skill{Python}{good}
  \end{skillhead}
  \begin{skillhead}{CAE software}
    \skill{OpenFOAM}{very good}
    \skill{Ansys Mechanical}{good}
    \skill{Ansys Fluent}{good}
    \skill{ParaView}{very good}
  \end{skillhead}
  \begin{skillhead}{CAD software}
    \skill{Siemens NX}{very good}
    \skill{Autodesk Inventor}{good}
    \skill{Solidworks}{intermediate}
  \end{skillhead}
  \begin{skillhead}{Miscellaneous}
    \skill{\LaTeX}{very good}
    % \skill{Microsoft Office}{good}
    \skill{Linux-based systems}{intermediate}
  \end{skillhead}
  \vspace{0.3em}
\end{cvsection}

\begin{cvsection}{Languages}
  \skill{English}{very good (level C1)}
  \skill{German}{intermediate (level B1)}
  \skill{Polish}{native}
  \vspace{0.3em}
\end{cvsection}

\begin{cvsection}{Personal interests}
  Sailing, science-fiction literature
\end{cvsection}
}


  
\end{document}
